\documentclass{urticle}

\usepackage[colorlinks=true]{hyperref}

\usepackage[maxbibnames=99]{biblatex}
%\bibliographystyle{gost780s}
\bibliography{lib}

\begin{document}

\begin{titlepage}

    \begin{center}
        \large МОСКОВСКИЙ ФИЗИКО-ТЕХНИЧЕСКИЙ ИНСТИТУТ
        \vspace{2cm}\\
        \textsc{Программа экзамена}
        \vspace{8cm}\\
        \LARGE Государственный экзамен по математике
        \vspace{3cm}\\
        \large Выполнил работу: Никита \textsc{Мокров}
    \end{center}

    \vfill
    \center Долгопрудный, 2017г.
\end{titlepage}

\tableofcontents
\newpage

\section{Введение}
\label{Intro}
    В данном документе собран материал для подготовки к государственному экзамену по математике в МФТИ. Курс математического анализа базируется на книгах Г.Н. Яковлева~\ref{Yakovlev} и лекциях Р.Н. Карасева~\ref{Karasev}. Курс линейной алгебры и аналитической геометрии основан на лекциях и методичках П.А. Кожевникова~\ref{Kozhevnikov}. Курс дифференциальных уравнений собран из книг~\ref{Diff1}.

\section{Математический анализ}
\label{MathAnalysis}

\subsection{Теорема Больцана-Вейрштрасса и критерий Коши сходимости числовой послежовательности}

\subsection{Ограниченность функий, непрерывной на отрезке, достижение точных верхней и нижней граней}




\section{Линейная алгебра}
\label{LinAlgebra}



\end{document}
