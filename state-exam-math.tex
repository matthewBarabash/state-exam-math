\documentclass{urticle}

\usepackage[colorlinks=true]{hyperref}


\usepackage{amsthm}
\newtheorem{theorem}{Теорема}
\newtheorem{lemma}{Лемма}
\newtheorem{consectary}{Cледствие}
\theoremstyle{definition}
\newtheorem{definition}{Определение}
\newtheorem{proposal}{Предложение}
\newtheorem{remark}{Замечание}

\newcommand{\prf}[1]{\hspace{0.3cm}$\triangleright$ \hspace{0.2cm} {#1} \hfill $\blacksquare$ }

\begin{document}

\begin{titlepage}

    \begin{center}
        \large МОСКОВСКИЙ ФИЗИКО-ТЕХНИЧЕСКИЙ ИНСТИТУТ
        \vspace{2cm}\\
        \textsc{Программа экзамена}
        \vspace{8cm}\\
        \LARGE Государственный экзамен по математике
        \vspace{3cm}\\
        \large Выполнил работу: Никита \textsc{Мокров}
    \end{center}

    \vfill
    \center Долгопрудный, 2017г.
\end{titlepage}

\tableofcontents
\newpage

\section*{Введение}
\label{Intro}
    В данном документе собран материал для подготовки к государственному экзамену по математике в МФТИ. Курс математического анализа базируется на книгах Г.Н. Яковлева~\ref{Yakovlev} и лекциях Р.Н. Карасева~\ref{Karasev}. Курс линейной алгебры и аналитической геометрии основан на лекциях и методичках П.А. Кожевникова~\ref{Kozhevnikov}. Курс дифференциальных уравнений собран из книг~\ref{Diff1}.

\section{Математический анализ}
\label{MathAnalysis}

\subsection{Теорема Больцана-Вейрштрасса и критерий Коши сходимости числовой последовательности.}
    
    \begin{definition}
    \label{def:PartialLimit}
        Предел любой подпоследотвальности даной последовательности называют \textit{частичным пределом}
    \end{definition}    
    
    \begin{theorem}
    \label{th:Bol-Ver}
        Любая ограниченная последовательность имеет хотя бы один частичный предел.
    \end{theorem}
    \prf{$\exists a, b; \forall n: a < x_n < b$. Тогда построим последовательность вложенных отрезков $[a_k, b_k], k\in\mathbb{N}$ путем деления отрезка попалам, начиная с $[a, b]$, и выбирая на каждой итерации правый отрезок, если он содержит бесконечное число членов последовательности в нем, и левый в противном случае. Причем, 
    \begin{center}
        $\displaystyle \lim_{k\to\infty}(b_k - a_k) = \lim_{k\to\infty} \frac{b-a}{2^k} = 0$  
    \end{center}        
    Следовательно, по теореме Кантора они имеют одну общую точку
    \begin{center}
        $\displaystyle c = \lim_{k\to\infty}a_k = \lim_{k\to\infty}b_k$
    \end{center}
    Рассмотрев последовательно каждый отрезок и выбрав в нем один член последоватльности, по номеру больший, чем предыдуший, мы смоделировали подпоследовательность, которая будет стремиться к $c$ по теореме о трех последовательностях (о двух милиционерах).
    }
    
    \begin{definition}
    \label{def:UpLowLimit}    
        Наибольший (наименьший) частичный предел последовательности называется ее \textit{верхним (нижним) пределом}.
    \end{definition}
    
    \begin{consectary}
    \label{th:UpLowLimit}    
        Любая ограниченная последовательность имеет верхний и нижний предел.
    \end{consectary}
    \prf{Число $c$ из теоремы~\ref{th:Bol-Ver} очевидно является верхним пределом. Если при построении вложенных отрезков первым выбирать левый, то получим нижний предел. }
    
    \begin{lemma}
    \label{lem:Koshi}
        Если последоватльность $\{x_n\}$ сходится, то она удовлетворяет условию Коши:
        \begin{equation}
        \label{IfKoshi}
            \forall\varepsilon  \quad \exists N_\varepsilon: \forall n,m \geq N_\varepsilon \quad |x_n - x_m| < \varepsilon.
        \end{equation}
    \end{lemma}
    \prf{Пусть $\displaystyle \lim_{n\to\infty}x_n = x_0$ т.е. $\forall\varepsilon  \quad \exists N_\varepsilon: \forall n \geq N_\varepsilon \quad |x_n - x_0| < \dfrac{\varepsilon}{2}$. Значит, для $\forall n \geq N_\varepsilon$ и для $\forall m \geq N_\varepsilon$:
    \begin{center}
        $|x_n - x_m| \leq |x_n - x_0| + |x_m - x_0| \leq \dfrac{\varepsilon}{2} + \dfrac{\varepsilon}{2} \leq \varepsilon $
    \end{center}
    }
        
    \begin{theorem}
    \label{th:IfKoshi}
        Если числовая последовательность удовлетворяет условию Коши, то она имеет конечный предел.
    \end{theorem}
    \prf{Положив $\varepsilon = 1$ и $m = N_1$, очевидно следует ограниченность последовательности, удовлетворяющая условю Коши. Тогда по теореме~\ref{th:Bol-Ver} существует сходящаяся подпоследовательность. Пусть $\displaystyle \lim_{k\to\infty}x_{n_k} = x_0$. Тогда из очевидного неравенства $|x_n - x_0| \leq |x_n - x_{n_p}| + |x_{n_p} - x_0|$ следует, что $\displaystyle \lim_{n\to\infty}x_n = x_0$.}
    
    \begin{consectary}[Критерий Коши]
    \label{th:KrKoshi}
        Числовая последовтельность сходится тогда и только тогда, когда она удовлетворяет условию Коши.
    \end{consectary}

\subsection{Ограниченность функций, непрерывной на отрезке, достижение точных верхней и нижней граней.}

    \begin{definition}    
    \label{def:SupInf}
        \textit{Точной верхней гранью множетсва X} ($supX$) называется такое число M:
        \begin{enumerate}
        \item $\forall x \in X \quad x \leq M$;
        \item $\forall M' < M \quad \exists x_{M'} \in X: x_{M'} > M'$.
        \end{enumerate}
        По аналогии вводится $infX$.
    \end{definition}
    
    \begin{theorem}
    \label{th:OnlySupInf}
        Любое множество действительных чисел может иметь лишь одну точную верхнюю (нижнюю) грань.
    \end{theorem}
    \prf{Доказывая от противного, получим противоречие со вторым условием в опр.~\ref{def:SupInf}.}
    
    \begin{theorem}
    \label{th:ExistSupInf}    
        У любого непустого множества действительных чисел ограниченного сверху (снизу), существует точная верхняя (нижняя) грань, являющаяся действительным числом. 
    \end{theorem}
    \prf{ Пусть $a \in X$ и $\exists b: x \leq b \quad \forall x \in X$. Тогда отрезок $[a, b]$ содержит хотя бы один элемент из $X$. Рассмотрим нетривиальный случай $a<b$. Построим последовательность вложенных отрезков $[a_n, b_n]$, по аналогии из теоремы~\ref{th:Bol-Ver}, таких что 
    \begin{enumerate}
    \item $x \leq b_n \quad \forall x \in X\: \forall n \in \mathbb{N}$;
    \item $\forall a_n \quad \exists x_n \in X: \quad x_n > a_n$.
    \end{enumerate}
    Значит мы получим точку $c$, к которой стягиваются отрезки и которая по определению будет являться точной верхней гранью. Аналогично доказаывается существование точной нижней грани. }

\subsection{Теоремы о промежуточных значениях непрерывной функции.}
\subsection{Теоремы о среднем Ролля, Лагранжа и Коши для дифференцируемых функций.}
\subsection{Формула Тейлора с остатоным членом в форме Пеано или Лагранжа.}
\subsection{Исследование функции одной переменной при помощи первой и второй производныз на монотонность, локальные экстремумы, выпуклость. Необходимые и достаточные условия.}
\subsection{Теорема о равномерной непрерывности функции, непрерывной на компакте.}
\subsection{Достаточные условия дифференцируемости функции нескольких переменных.}
\subsection{Теорема о неявной функции, заданной одним уравнением.}
\subsection{Экстремумы функций нескольких переменных. Необходимые и достаточные условия.}
\subsection{Свойства интеграла с переменным верхним пределом. Формула Ньютона-Лейбница.}
\subsection{Равномерная сходимость функциональных последовательностей и рядов. Непрерывность, интегрируемость и дифференцируемость суммы функционального ряда.}
\subsection{Степенные ряды. Радиусы сходимости. Бесконечная дифференцируемость суммы степенного ряда. Ряд Тейлора.}
\subsection{Формула Грина. Потенциальные векторные поля на плоскости.}
\subsection{Формула Остроградского-Гаусса. Соленоидальные векторные поля.}
\subsection{Формула Стокса.}
\subsection{Достаточные условия сходимости тригонометрического ряда Фурье в точке.}
\subsection{Достаточные условия равномерной сходимости тригонометрического ряда Фурье.}
\subsection{Непрерывность преобразования Фурье абсолютно интегрируемой функции. Преобразование Фурье произовдной и произовдная преобразования Фурье.}

\section{Линейная алгебра}
\label{LinAlgebra}

\subsection{Углы между прямыми и плоскостями. Формулы расстония от точки до прмямой и плоскости, между прямыми в пространстве.}
\subsection{Общее решение системы линейных алгебраическиx уравнений. Теорема Кронекера-Капелли.}
\subsection{Линейное отображение конечномерных линейных пространств, его матрица. Свойства собственныз векторов и собственных значений линейных преобразований.}
\subsection{Самосопряженныые преобразования евклидовых пространств, свойсвта их собственных значений и собственных векторов.}
\subsection{Приведение квадратичных форм в линейном пространстве к каноническому виду.}
\subsection{Положительно определенные квадратичные форму. Критерий Сильвестра.}

\section{Дифференициальные уравнения}
\label{DiffEq}


\bibliography{lib}
\end{document}
